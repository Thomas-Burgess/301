\documentclass{article}
\usepackage[utf8]{inputenc}
\usepackage[margin=1in]{geometry}
\usepackage{amsmath}
\usepackage{amssymb}
\usepackage{graphicx}
\usepackage{xcolor}

\title{Lab 1 Example}
\author{} % can be blank
\date{}   % can be blank

\begin{document}

\maketitle

\section{Student Information}
\begin{enumerate}
    \item Thomas Burgess
    \item Donnelly
\end{enumerate}

\section{Natural Numbers}
The set of natural numbers is defined to be:
\begin{equation}
\mathbb{N} = \{1,2,3,...\}.
\end{equation}

\section{Subscripts and Superscripts}
Let $X$ be a $10$ $\times$ $10$ matrix.
\subsection{Subscripts}
$X_i$ denotes the $i$th column of $X$, while $X_{i_j}$ denotes the $(i,j)$th entry
\subsection{Superscripts}
$X^T$ means $X$ transposed. $X^T_i$ is the $i$th column of $X^T$; in other words, the $i$th row of X.

\section{Equation Fun}
\begin{itemize}
    \item Here's the \emph{quadratic equation}, for no particular reason:
    \begin{equation}
    x = \frac{-b \pm \sqrt{b^2 - 4ac}}{2a}
    \end{equation}
    
\end{itemize}

\begin{itemize}
    \item Here's \textbf{Euler's identity}, for no particular reason:
    \begin{equation}
    e^{i\pi} + 1 = 0.
    \end{equation}
\end{itemize}
\end{document}
